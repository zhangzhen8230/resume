% !TEX program = xelatex
% This is my resume
% Chinese translation
% by ice1000

\documentclass{resume}

\usepackage{lastpage}
\usepackage{fancyhdr}
\usepackage{linespacing_fix} % disable extra space before next section
\usepackage[fallback]{xeCJK}

%% \setmainfont[]{SimSun}
%% \setCJKfallbackfamilyfont{rm}{HAN NOM B}
\setCJKmainfont[BoldFont=SimHei,ItalicFont=Arial]{SimSun}
%% \renewcommand{\thepage}{\Chinese{page}}

\begin{document}
\pagestyle{fancy}
\fancyhf{}
\renewcommand\headrulewidth{0pt}


\name{张真}
\centerline{求职意向:自然语言处理|算法工程师}
\basicInfo{
  \email{zhangzhen82301@gmail.com} \textperiodcentered\ 
  \phone{(+86) 131-6141-1563} \textperiodcentered\ 
  \github[zhangzhen8230]{https://github.com/zhangzhen8230/}
  % \linkedin[billryan8]{https://www.linkedin.com/in/billryan8}
}
\section{\faUsers\ 实习经历}

\datedsubsection{\textbf{腾讯科技}, 北京}{2018.8 -- 2018.11}
{医疗诊断预测,论文实验与撰写}
\begin{onehalfspacing}
\begin{itemize}
  \item 负责相关论文的调研与复现,包括doctorAI,retain,dipole.
  \item 负责收集不同数据集,并进行实验,包括数据集MIMICIII,MIMICII,CMS.
  \item 分析标签编码的结构,在模型中添加引入层次结构知识,评价指标top30召回率从56.9\% 提升到62.3 \%
  \item 尝试添加时序信息,评价指标top30召回率从基础的56.9 \%,提升到58.2\%.
\end{itemize}
\end{onehalfspacing}
\vspace{-1ex}
\section{\faGithubAlt\ 课题项目}
\datedsubsection{\textbf{流式文档排版格式的智能化分析与优化方法}}{2017.9--2018.12}
{国家自然科学基金,61672105}
\begin{onehalfspacing}
\begin{itemize}
  \item \textbf{语料库的收集.}确定标签集合与语料类别,编写标注插件,制定相应的xml scheme,最终获得5000篇标注语料.
  \item \textbf{文档特征的提取.}提取格式特征18种,内容特征6种以及doc2vec编码的语义特征,将格式特征与内容特征进行特征处理.
  \item \textbf{模型搭建.}针对文档构件类别不平衡问题,采用规则与神经网络相结合的方法;对文档类型多种多样等要求,构造迁移学习模型.针对文档构件的上下文关系,使用LSTM学习。
    \item \textbf{系统集成.}完成文档结构自动识别系统,整合了上几届的代码。
\end{itemize}
\end{onehalfspacing}
% Reference Test
%\datedsubsection{\textbf{Paper Title\cite{zaharia2012resilient}}}{May. 2015}
%An xxx optimized for xxx\cite{verma2015large}
%\begin{itemize}
%  \item main contribution
%\end{itemize}

%% \section{\faHeartO\ 成就}
%% \datedline{}{Aug. 2017}

\vspace{-1ex}
\section{\faSitemap\ 竞赛经历}
\datedsubsection{\textbf{天池对话领域竞赛}}{2018.6--2018.8}
{“阿里小蜜机器人跨语言短文本匹配算法竞赛”,队长}
\begin{onehalfspacing}
\begin{itemize}
	\item 针对文本匹配目标,构建跨语言的短文本匹配模型。从统计特征,NLP特征和图特征三个方面构造特征工程 \item 其中针对于跨语言问题,采用auto-encoder模型编码句子向量。
	\item 最后采用Stack方式进行模型融合,最终在1027个队伍中,获得了初赛第5,复赛第8的成绩。

\end{itemize}
\end{onehalfspacing}
\begin{onehalfspacing}
\datedsubsection{\textbf{ CCF情感分析竞赛,32/1186}}{2018.2--2018.4}
{景区口碑评价分值预测竞赛,队长}
\begin{itemize}
	\item 针对旅游景点的评论,构建评论打分模型。使用word2vec,doc2vec生成句子表示。
	\item 使用lstm,textcnn等模型学习文档特征表示。
	\item 最后采用blend方式进行模型融合,最终在1186个队伍中,获得了初赛第13,复赛第32的成绩。
\end{itemize}
\end{onehalfspacing}
\vspace{-1ex}
\section{\faCogs\ 技能}
\begin{onehalfspacing}
\begin{itemize}[parsep=0.25ex]
  \item \textbf{编程语言}: 熟悉python C\#,java.
  % compiler theories
  \item \textbf{深度学习}:
    两年nlp经验,熟悉tensorflow与keras框架,熟悉常见深度学习模型原理.

  % language Kotlin
  \item \textbf{英语能力}: CET-6 494分,良好的英语阅读能力.
   \item \textbf{组织能力}:北京市优秀班集体(申请负责人),
   优秀团支部,班集体(连续四年,负责人), 优秀团干部(3次)			
    \item \textbf{学业情况}: 校级二等奖学金(2次),三等奖学金(3次),三好学生

\end{itemize}
\end{onehalfspacing}
% \section{\faHeartO\ Honors and Awards}
% \datedline{\textit{\nth{1} Prize}, Award on xxx }{Jun. 2013}
% \datedline{Other awards}{2015}
\vspace{-1ex}
\section{\faGraduationCap\ 教育经历}
\datedsubsection{\textbf{北京信息科技大学}\          计算机学院,硕士}{2016.9 -- 现在}
专业:数字文化传播,预计毕业日期:2019.6
\datedsubsection{\textbf{北京信息科技大学}\          计算机学院,本科}{2012.9 -- 2016.6}
专业:软件工程


%% Reference
%\newpage
%\bibliographystyle{IEEETran}
%\bibliography{mycite}
\end{document}
